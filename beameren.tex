% file name: beameren.tex
% http://www.mamicode.com/info-detail-247246.html
% 原文地址:http://www.cnblogs.com/jinliangjiuzhuang/p/4014459.html
%
% 于海军,北京大学,

\documentclass[11pt]{beamer}


%% Copyright 2004 by Yu Haijun <yhj@math.pku.edu.cn>
%%
%% 这个文件的目的使为了给大家提供一个使用beamer的快速模板

%%% Local Variables:
%%% mode: latex
%%% TeX-master: t
%%% End:

\mode<article> % 仅应用于article版本
{
  \usepackage{beamerbasearticle}
  \usepackage{fullpage}
  \usepackage{hyperref}
}

%% 下面的包控制beamer的风格,可以根据自己的爱好修改
\usepackage{beamerthemesplit}   % 使用split风格
\usepackage{beamerthemeshadow}  % 使用shadow风格
%% 这些包是可能会用到的,不必修改
\usepackage{pgf,pgfarrows,pgfnodes,pgfautomata,pgfheaps}
\usepackage{amsmath,amssymb}
\usepackage{graphics}
\usepackage{multimedia}

%% 下面的代码用来读入Logo图象
\pgfdeclaremask{logomask}{pku-tower-mask}
\pgfdeclareimage[mask=logomask,height=1.5cm]{logo}{pku-tower}

\pgfdeclaremask{beidamask}{beida-mark-mask}
\pgfdeclareimage[mask=beidamask,height=0.25cm]{beida}{beida-mark}

\pgfdeclaremask{titlemask}{pku-lake2-mask}
\pgfdeclareimage[mask=titlemask,height=2.5cm]{title}{pku-lake2}
\logo{\vbox{\hbox{\hfill\pgfuseimage{logo}}}} %设置logo图标

%% 定义一些自选的模板,包括背景、图标、导航条和页脚等,修改要慎重
\beamertemplateshadingbackground{red!10}{structure!10}
%\beamertemplatesolidbackgroundcolor{white!90!blue}
\beamertemplatetransparentcovereddynamic
\beamertemplateballitem
\beamertemplatenumberedballsectiontoc
%\beamertemplatelargetitlepage
\beamertemplateboldpartpage

\makeatletter
\usefoottemplate{ %重新定义页脚,加入作者,单位,单位图标,和文档标题
  \vbox{\tiny%
    \hbox{%
      \setbox\beamer@linebox=\hbox to\paperwidth{%
        \hbox to.5\paperwidth{\hfill\tiny\color{white}\textbf{\insertshortauthor\quad\insertshortinstitute}\hskip.1cm\lower 0.35em\hbox{\pgfuseimage{beida}}\hskip.3cm}%
        \hbox to.5\paperwidth{\hskip.3cm\tiny\color{white}\textbf{\insertshorttitle}\hfill}\hfill}%
      \ht\beamer@linebox=2.625ex%
      \dp\beamer@linebox=0pt%
      \setbox\beamer@linebox=\vbox{\box\beamer@linebox\vskip1.125ex}%
      \color{structure}\hskip-\Gm@lmargin\vrule width.5\paperwidth
      height\ht\beamer@linebox\color{structure!70}\vrule width.5\paperwidth
      height\ht\beamer@linebox\hskip-\paperwidth%
      \hbox{\box\beamer@linebox\hfill}\hfill\hskip-\Gm@rmargin}
  }
}
\makeatother


%%
%% 自己的预定义命令和宏放在这里
%%

%%
%% 填写作者,单位,日期,标题等文档信息
%%
\title{Make~Slides Using~Beamer}
\subtitle{Beamer - The~\LaTeX~Document~Class}
\author[Yu Haijun]{Yu Haijun}
\institute[DSEC, CCSE, at PKU]{
  Department of Science and Engineering Computing   School of Mathematics School   Peking University
}
\date[ND,2004]{Development in National Day, 2004}
\subject{Computer Tools, TeX, Slide}
\titlegraphic{\pgfuseimage{title}}

\AtBeginSection[]{ % 在每个Section前都会加入的Frame
  \frame<handout:0>{
    \frametitle{Outline}
    \tableofcontents[current,currentsubsection]
  }
}

%%
%% 文档从这里正式开始
%%   使用\part,\section,\subsection等命令组织文档结构
%%   使用\frame命令制作幻灯片
%%

\begin{document}

\defverbatim\beamerEX{
\begin{verbatim}
\documentclass{beamer}
\usepackage{beamerthemesplit}
\title{Example Presentation Created with Beamer}
\author{Till Tantau}
\date{\today}
\begin{document}
\frame{\titlepage}
\section*{Outline}
\frame{\tableofcontents}
\section{Introduction}
\subsection{Overview of the Beamer Class}
\frame {
   \frametitle{Features of the Beamer Class}
   \begin{itemize}
       \item<1-> Normal LaTeX class.
       \item<2-> Easy overlays.
       \item<3-> No external programs needed.
   \end{itemize}
}
\end{document}
\end{verbatim}
}

\frame[plain]{\titlepage} % 产生主题页,plain选项表示不显示页眉页脚等内容

%% \section<presentation>*{Outline}   % 带*号表示不加到目录中
%% \frame{
%%   \nameslide{outline}
%%   \frametitle{Outline}
%%   \tableofcontents[pausesections]
%%   \note{At most 1 minute for the outline}
%% }

\part{Slides Tools}
\frame{\partpage}

\section{Tools Like Powerpoint}
\frame[<+->]{
  \frametitle{Tools Like Powerpoint}
  \begin{itemize}
  \item Advantage
    \begin{enumerate}
    \item What you see is what you get
    \item All done in one software
    \item Easy to learn
    \end{enumerate}
  \item Disadvantage
    \begin{enumerate}
    \item Commonly the software is not free
    \item It depand on the software
    \item It‘s hard for much formula
    \end{enumerate}
  \end{itemize}
}
\section{\TeX Tools}

\frame[<+-| alert@+>]{
  \frametitle{TeX Tools}
  \begin{enumerate}
  \item Base PDF file
  \item Deal with mathematic formula easily
  \item Professional typeset
  \item Plain text, easy to reuse
  \end{enumerate}
}
\frame[<+->]{
  \frametitle{\TeX\ Slide Tools}
  \begin{description}
  \item[Beamer] A~standard~\LaTeX\ ~Document~ class,     Need~ no other post progress program     Work with other \LaTeX\ packages smoothly
  \item[foiltex] Work with most of the~available~\LaTeX\ ~commands~and~environments     Use Macro $\backslash$Mylogo put some graphic as the logo
   \item[prosper] Automatically generated table of contents, Portrait slides support      and possible to include notes in your presentation
  \item[pdfscreen] Create document both fit to read in computer and for print
% \item[seminar] A simple \LaTeX style designed for seminar presentations.
  \item[TeXPower] A \LaTeX\ style \textsl{texpower.sty}
  \end{description}
}


\part{Guidelines on Making Slides}
\frame{\partpage}

\section[What to Put on a Frame]{Guidelines on What to Put on a Frame}
\frame{
  \frametitle{\secname}
  \begin{enumerate}
  \item A frame with too little on it is better than a frame with too much on
    it.
  \item Do not assume that everyone in the audience is an expert on the
    subject matter.
  \item Nerver put anything on a slide that you are not going to explain
    during the talk.
  \item Keep it simple.
  \end{enumerate}
  }

\section[Titles]{Guidelines on Titles}

\frame{
  \frametitle{\secname}
  \begin{enumerate}
  \item Put a title on each frame
  \item The title should really explain things.
  \item Idealy, titles on consecutive frames should ``tell a story‘‘ all by
    themeselves.
  \item In English, you should \textsl{either always} capitalize all words in
    frame title except for words like ``a‘‘ or ``the‘‘(as in a title)
    \textsl{or} you \textsl{always} use the normal lowercase letters.
  \item In English, the title of the whole document should be capitalized,
    regardless of whether you capitalize anything else.
  \end{enumerate}
}

\section[Body Text]{Guidelines on the Body Text}
\frame{
  \frametitle{\secname}
  \begin{enumerate}
  \item Never use a smaller font size to ``fit more on a frame‘‘
  \item Prefer enumerations and itemize environment over plain text. Do not
    use long sentences.
  \item Do not hyphenate words. If absolutely necessary, hyphenate words ``by
    hand‘‘, using the command $\backslash$-
  \item Beak lines ``by hand‘‘ using the command $\backslash\backslash$. Do
    not rely on automatic line breaking.
  \item Text and numbers in figures should have the \textsl{same size} as
    normal text. Illegible numbers on axes usually ruin a chart and its
    message.
  \end{enumerate}
}

\section[Graphics]{Guidelines on Graphics}
\frame{
  \frametitle{\secname}
  \begin{enumerate}
  \item Put (at least) one graphic on each slide, whenever possible.
  \item Usually, place graphics to the left of the text
  \item Graphics should have the smae typographic parameters as the text
  \item While bitmap graphics, like photos, can be much more colorful than the
    rest of the text, vector graphics should follow the same ``color logic‘‘
    as the main text (like black==normal lines, red==hilighted parts,
    green==examples, blue==structure)
  \item Like text, you should explain everything that is shown on a graphic
  \item
  \end{enumerate}
}

\section[Colors]{Guidelines on Colors}
\frame{
  \frametitle{\secname}
  \begin{enumerate}
  \item Use colors sparsely. The prepared themes are already quite colorful
  \item Becareful when using bright colors on white background,
    \textsl{especially} when using green.
  \item Maximize contrast.Normal text should be black on white or at least
    something very dark on something very bright.
  \item Background shadings decrease the legibility without increasing the
    information content.  Inverse video (bright text on dark background) can
    be a problem during presentations in bright environments since only a
    small precentage of the presentaion area is light up by the beamer.
    Inverse video is harder to reproduce on printouts and on trnasparencies.
  \end{enumerate}
}

\section[Animations and Special Errects]{Guidelines on Animations and Special Effects}
\frame{
  \frametitle{\secname}
  \begin{enumerate}
  \item Use animations to explain the dynamics of systems, algorithms, etc.
  \item Do not use animations just to attract the attention of your audience.
    This often distracts attention away from the main topic of the slide
  \item Do not use distracting special effects like ``dissolving‘‘ slides
    unless you have a very good reason for using them.
  \end{enumerate}
}


\part{Make Slides Using Beamer Class}
\frame{\partpage}

\section{The Features of Beamer}
\frame{
  \frametitle{\secname}
  Beamer is standard \LaTeX document class, it has following features:   \begin{enumerate}[<+-| alert@+>]
  \item Only \LaTeX and pdflatex is need
  \item Retain section structures
  \item Themes and content and indenpent
  \item Easy to use
  \end{enumerate}
}

\section{Installation}
\frame{
  \frametitle{\secname}
  \begin{enumerate}
  \item First, copy xcolor, pgf, beamer files in preper texmf directory
  \item Second, Rehash the \TeX configuration
  \end{enumerate}
}

\section{Workflow}
\frame[containsverbatim]{
  \frametitle{\secname}
  \begin{enumerate}
  \item Create the structure, using \verb|\part| \verb|\section| \verb|\subsection|
  \item Add Frames and Overlays, using \verb|\frame|
  \item Apply Themes and templates, using \verb|\usepackage|
  \end{enumerate}
}


\frame{ % 由于对包含了verbatim的frame换页有问题,所以我们手工加入一页
  \frametitle{A Beginning File of Beamer}
  \beamerEX
}

\part{Step by Step}
\frame{\partpage}

\section{Frames and Overlays}
\subsection{Overlays}
\frame{
  \frametitle{Overlays}
  \textbf{Commands of Overlays:}
  \begin{itemize}[<+-| alert@+>]
  \item $\backslash$onslide$<$\textsl{slide-list}$>$
  \item $\backslash$FromeSlide$<$\textsl{slide-number}$>$
  \item $\backslash$only$<$\textsl{slide-number}$>$
  \item slide specifity after other command, e.g.     \hskip 0.5cm $\backslash$textbf$<$2$>$
  \end{itemize}
}

\frame{
  \frametitle{Action of Overlays}
  \begin{enumerate}
  \item<1-| alert@1> alert
  \item<2-| uncover@2> uncover
  \item<3-| only@3> only
  \item<4-| visible@4> visible
  \item<5-| invisible@5> invisible
  \end{enumerate}
}

\beamerdefaultoverlayspecification{<+-| alert@+>}
\frame[<+->]{
  \frametitle{Action of Overlays II}
  \begin{itemize}
  \item You can specific action indendent, e.g     $\backslash action<\mbox{action-specification}>\{<text>\}$
  \item Set the default action using following command     $\backslash beamerdefaultoverlayspecification\{<deault-overlay-specification>\}$
  \end{itemize}
}

\subsection{Frames}
%%\frame<overlay-specification>[<default-overlay-specification>][<options>]{<frame text>}
\frame<presentation>[<+->]{
  \frametitle{Frames}
  \begin{theorem}
    $A=B$
  \end{theorem}
  \begin{proof}
    \begin{itemize}
    \item Clearly, $$A=\int_0^\infty e^{x^2}\,dx$$
    \item As show earlier, $$\int_0^\infty e^{x^2}\,dx=B$$
    \item Thus $A=B$
    \end{itemize}
  \end{proof}
}
\frame[allowframebreaks]{
  \frametitle{Frames II}
  \textbf{Options of Frame}
  \begin{itemize}
  \item allowdisplaybreaks=$<$\textsl{break-desirability}$>$
  \item allowframebreaks=$<$\textsl{fraction}$>$      Note: Frame break will has no overlays effects!
  \item b,c,t -- vertically aligned at bottom/center/top
  \item containsverbatim      Only one slide of the frame is typeset!
  \item label=$<$\textsl{name}$>$
  \item plain -- cause the headlines, footlines and sidebars to be suppressed
  \item shrink=$<$\textsl{minimum-shrink-percentage}$>$
  \item squeeze -- squeeze vertical spaces.
  \end{itemize}
}

\frame{
  \frametitle{Commponents of a Frame}
  \begin{itemize}
  \item a headline
  \item a footline
  \item a left sidebar
  \item a right sidebar
  \item navigation symbols
  \item a logo
  \item a frame title, and
  \item some frame contents
  \end{itemize}
}
\frame[<+->]
{
  \frametitle{Hyperlink and Navigation Bars}
  \begin{itemize}
  \item Use \textbf{hypertarget} add hyper link.     $\backslash hypertarget<\mbox{overlay-specification}>\{\mbox{target-name}\{\mbox{text}\}$
    \hyperlink{jumptosecond}{\beamergotobutton{Jump to second slide}}
    \hypertarget<2>{jumptosecond}{}
  \item \beamerbutton{beamerbutton}
  \item \beamerskipbutton{beamerskipbutton}
  \item \beamerreturnbutton{beamerreturnbutton}
  \end{itemize}
}

\section{Color Management}
\frame[<+->]{
  \frametitle{\secname}
  \begin{itemize}
  \item Change the main color of navigation and title bar      $\backslash\mbox{documentclass[red]}\{beamer\}$
  \item Change the average background color      $\backslash beamersetaveragebackground\{\mbox{red!}10\}$
  \item Set how to render overlay covered text.      $\backslash beamersetunconvermixins\{\mbox{not-yet-list}\}\{\mbox{once-more-list}\}$
  \item Set on which slides covered text should have which opaqueness.     $\backslash opaqueness<\mbox{overlay-specification}>\{\mbox{percentage-of-opaqueness}\}$
  \end{itemize}
}

\section{Graphics, Animations, sounds, and Slide Transitions }
\frame{
  \frametitle{\secname}
  \begin{itemize}
  \item Graphics      \includegraphics<1>{pku-logo.pdf}
    \includegraphics<2>{pku-tower.pdf}
    \pgfuseimage<3>{logo}
  \item Animations      \movie[externalviewer,label=mymovie,width=1in,height=0.8in,poster]{}{movie.avi}
%    \hyperlinkmovie[play]{mymovie}{Play}
  \item Sound
    \movie[externalviewer,autostart]{Here‘s some music}{turky.mp3}
  \end{itemize}
}

\frame[<+->]{
  \frametitle{Slide Transitions}
  \begin{enumerate}
  \item Horizontal blinds
    \transblindshorizontal<1>
  \item Vertical blinds
    \transblindsvertical<2>
  \item Moving to the center from all four sides
    \transboxin<3>
  \item Moving from the center to four sides
    \transboxout<4>
  \item Dissolve
    \transdissolve<5>
  \item Glitter
    \transglitter<6>
  \item Split verticalin
    \transsplitverticalin<7>
  \item Split verticalout
    \transsplitverticalout<8>
  \item wipe
    \transwipe<9>
  \item transduration
    \transduration<10>{1}
  \end{enumerate}
}


\section{Customization}

%%
%% reference here
%%

\section<presentation>*{Reference}

\frame{
  \transdissolve
  \frametitle<presentation>{Reference}

  \beamertemplatebookbibitems

  \begin{thebibliography}{10}

  \bibitem{TeX}
    Wynter Snow
    \newblock {\em \TeX for the Beginner}.
    \newblock Addison-Wesley Publishing Company, 1992.
    \pause

  \beamertemplatearticlebibitems

  \bibitem{userguide}
    Till Tantau
    \newblock User‘s Guide to the Beamer Class, Version 2.20
    \newblock {\em http://latex-beamer.sourceforge.net}, April 19,2004
    \pause


  \end{thebibliography}
}



%%
%% 附录之后的东西将不在正文里出现
%%

\appendix

\section{Tips and (Dirty) Tricks}

%%
%% 最后加入一个显示Thank you!!!字样的动画
%%

\newcount\opaqueness
\plainframe{
  \itshape
  \animate<1-10>
  \Large

  \only<1-10>{
  \animatevalue<1-10>{\opaqueness}{10}{100}
  \begin{colormixin}{\the\opaqueness!averagebackgroundcolor}
    \begin{centering}
      \Huge Thank You!!!\par
    \end{centering}
  \end{colormixin}
  }
}

\end{document}

