% file name: beamercn.tex
% website: http://www.mamicode.com/info-detail-247246.html
%

\documentclass[CJK]{beamer}

%% Copyright 2004 by Yu Haijun <yhj@math.pku.edu.cn>
%%
%% 这个文件的目的使为了给大家提供一个使用beamer的快速模板

%%% Local Variables:
%%% mode: latex
%%% TeX-master: t
%%% End:

\mode<article> % 仅应用于article版本
{
  \usepackage{beamerbasearticle}
  \usepackage{fullpage}
  \usepackage{hyperref}
}

%% 下面的包控制beamer的风格,可以根据自己的爱好修改
%\usepackage{beamerthemesplit}   % 使用split风格
\usepackage{beamerthemeshadow}  % 使用shadow风格
\usepackage[width=2cm,dark,tab]{beamerthemesidebar}


%%%%%%%%%%%%%%%%%%%%%%%%%%%%%%
\usepackage[utf8]{inputenc}
\usepackage{fontspec}
\usepackage{xeCJK}
\setCJKmainfont{WenQuanYi Micro Hei Mono}
%%%%%%%%%%%%%%%%%%%%%%%%%%%%%%

%% 这些包是可能会用到的,不必修改
\usepackage{CJK}
\usepackage{pgf,pgfarrows,pgfnodes,pgfautomata,pgfheaps}
\usepackage{amsmath,amssymb}
\usepackage{graphicx}
\usepackage{multimedia}

%% 下面的代码用来读入Logo图象
\pgfdeclaremask{logomask}{pku-tower-lake-mask}
\pgfdeclareimage[mask=logomask,width=1.8cm]{logo}{pku-tower-lake}

\pgfdeclaremask{beidamask}{beida-mark-mask}

\pgfdeclaremask{titlemask}{pku-lake2-mask}
\pgfdeclareimage[mask=titlemask,height=2.5cm]{title}{pku-lake2}
\logo{\vbox{\hbox{\pgfuseimage{logo}\hfill}}} %设置logo图标

%% 定义一些自选的模板,包括背景、图标、导航条和页脚等,修改要慎重
\beamertemplateshadingbackground{red!10}{structure!10}
%\beamertemplatesolidbackgroundcolor{white!90!blue}
\beamertemplatetransparentcovereddynamic
\beamertemplateballitem
\beamertemplatenumberedballsectiontoc
%\beamertemplatelargetitlepage
\beamertemplateboldpartpage

%% \makeatletter
%% \usefoottemplate{ %重新定义页脚,加入作者,单位,单位图标,和文档标题
%%   \vbox{\tiny%
%%     \hbox{%
%%       \setbox\beamer@linebox=\hbox to\paperwidth{%
%%         \hbox to.5\paperwidth{\hfill\tiny\color{white}\textbf{\insertshortauthor\quad\insertshortinstitute}\hskip.1cm\lower 0.35em\hbox{\pgfuseimage{beida}}\hskip.3cm}%
%%         \hbox to.5\paperwidth{\hskip.3cm\tiny\color{white}\textbf{\insertshorttitle}\hfill}\hfill}%
%%       \ht\beamer@linebox=2.625ex%
%%       \dp\beamer@linebox=0pt%
%%       \setbox\beamer@linebox=\vbox{\box\beamer@linebox\vskip1.125ex}%
%%       \color{structure}\hskip-\Gm@lmargin\vrule width.5\paperwidth
%%       height\ht\beamer@linebox\color{structure!70}\vrule width.5\paperwidth
%%       height\ht\beamer@linebox\hskip-\paperwidth%
%%       \hbox{\box\beamer@linebox\hfill}\hfill\hskip-\Gm@rmargin}
%%   }
%% }
%% \makeatother

%\beamersetuncovermixins% 改变遮挡部分在遮挡前后的透明程度
%{\opaqueness<1>{15}\opaqueness<2>{10}\opaqueness<3>{85}\opaqueness<4->{2}}%
%{\opaqueness<1->{15}}%



%%
%% 自己的预定义命令和宏放在这里
%%



%%
%% 文档从这里正式开始
%%   使用\part,\section,\subsection等命令组织文档结构
%%   使用\frame命令制作幻灯片
%%

\begin{document}
%\begin{CJK*}{GBK}{fs}


\defverbatim\beamerEX{
\begin{verbatim}
\documentclass{beamer}
\usepackage{beamerthemesplit}
\title{Example Presentation Created with Beamer}
\author{Till Tantau}
\date{\today}
\begin{document}
\frame{\titlepage}
\section*{Outline}
\frame{\tableofcontents}
\section{Introduction}
\subsection{Overview of the Beamer Class}
\frame {
   \frametitle{Features of the Beamer Class}
   \begin{itemize}
       \item<1-> Normal LaTeX class.
       \item<2-> Easy overlays.
       \item<3-> No external programs needed.
   \end{itemize}
}
\end{document}
\end{verbatim}
}


%%
%% 填写作者,单位,日期,标题等文档信息
%%
\title{使用Beamer制作Slide入门}
\subtitle{Beamer - The~\LaTeX~Document~Class}
\author{于海军}

\institute[DSEC, CCSE, at PKU]{
  Department of Science and Engineering Computing School of Mathematics School   Peking University
}
\date[ND,2004]{Development in National Day, 2004}
\subject{Computer Tools, TeX, Slide}
\titlegraphic{\pgfuseimage{title}}

%%
%% 定义框架页
%%
\AtBeginSection[]{                              % 在每个Section前都会加入的Frame
  \frame<handout:0>{
    \frametitle{框架}
    \tableofcontents[current,currentsubsection]
  }
}
\AtBeginSubsection[]                            % 在每个子段落之前
{
  \frame<handout:0>                             % handout:0 表示只在手稿中出现
  {
    \frametitle{框架}
    \tableofcontents[current,currentsubsection] % 显示在目录中加亮的当前章节
  }
}



\frame{\titlepage}

\section*{提纲}               % section后面加*表示不收录到目录中
\frame
{
  \frametitle{\secname}
  \tableofcontents            % 使用这个命令自动生成目录
}


\section{Slide的基本概念}

\frame{
  \frametitle{\secname}
  \begin{description}
  \item[何谓幻灯片]
    所谓Slide就是幻灯片的意思,是一种类似照片底片的透明胶片
  \item[幻灯片的作用]
    帮助演讲者向听众传达文字、图片甚至动画、声音等信息
  \item[幻灯片的优点]
    省去演讲者抄写时间     表达更准确,更直观     采用计算机,能传达更丰富的内容
  \end{description}
}

\subsection{Slide的实现方式}

\frame
{
  \frametitle{\subsecname}
  \begin{itemize}
  \item<+-> 所见即所得的工具
    \begin{enumerate}
    \item Powerpoint
    \item Magick Point
    \end{enumerate}
  \item<+-> 基于\TeX 和PDF的实现
    \begin{enumerate}[<+->]
    \item Beamer
    \item Foiltex
    \item ConTeXt
    \item prosper
    \item pdfscreen
    \end{enumerate}
  \end{itemize}

}

\subsection{\TeX Slide实现的特点}

\frame<1-4>[<+-| alert@+>][label=math]
{
  \frametitle{\subsecname}
  \begin{enumerate}
  \item 基于PDF文件格式,不需要专门放映工具,流通性强
  \item<+-| alert@2,5> 使用\TeX,处理数学公式方便
    \begin{equation}
      \label{eq1}
      \frac{1}{\sqrt{2\pi}}\int_{-\infty}^{\infty}e^{- {x^2} \over 2}\,dx = 1
    \end{equation}
    \only<5>{\hyperlink{jumptofifth}{\beamerreturnbutton{返回}}}
  \item 足够用的动态效果
  \item 纯文本文件,便于处理和传播
  \end{enumerate}
}

\section{Beamer文件的框架}

\subsection{逻辑和内容的独立}

\frame
{
  \frametitle{\secname}
  \begin{itemize}
  \item 使用part,section,subsection等命令组织逻辑结构
  \item 使用frame命令组织内容
  \end{itemize}
}

\subsection{内容和显示效果的独立}

\frame
{
  \frametitle{\subsecname}
  \begin{itemize}
  \item 使用themes,template,logo改变缺省风格
  \item 使用overlay选项控制临时效果
  \item 通过文档类选项控制输出格式等
  \end{itemize}
}

\section{Beamer效果演示}

\subsection{逐行显示的实现}

\frame
{
  \frametitle{\subsecname}
  \begin{itemize}[<+-|  alert@+>]
  \item 这一段在第一个Slide上显示,并被加亮
  \item 这一段在第二个Slide上显示,并被加亮
  \item 这一段在第三个Slide上显示,并被加亮
  \end{itemize}
}

\subsection{字体和色彩的演示}

\frame
{
  \frametitle{\subsecname}
  {\textbf<1> 1. \alt<1>{\CJKfamily{hei}这是黑体在第一张上}%可以为中文的字体变换定义一个宏
    {\CJKfamily{fs}这是黑体在第一张上} \\}
  {\textit<2> 2. \CJKfamily{song}这是斜体,在第二张上\\}
  {\color<3>[rgb]{1,0,0} 3. 这些文字是在第3张幻灯片上是红色的,其它是黑色的。\\}
  \only<4>{ 4. 仅在第四张出现。\\}
  \alert<4>{4.alert代表红色\\}
  \structure<5>{5. structure代表绿色\\}
  \alt<6>{6. 仅在第6张}{6. 在1-5张上}
}

\subsection{换页动态效果}

\frame[<+->]{
  \frametitle{\subsecname}
  \begin{enumerate}
  \item 水平出现效果
    \transblindshorizontal<1>
  \item 竖直出现效果
    \transblindsvertical<2>
  \item 从中心到四角
    \transboxin<3>
  \item 从四角到中心
    \transboxout<4>
  \item 溶解效果
    \transdissolve<5>
  \item Glitter
    \transglitter<6>
  \item 竖直撕开(向内)
    \transsplitverticalin<7>
  \item 竖直撕开(向外)
    \transsplitverticalout<8>
  \item 涂抹
    \transwipe<9>
  \item 渐出
    \transduration<10>{1}
  \end{enumerate}
}

\subsection{超级链接的实现}

\frame
{
  \frametitle{\subsecname}
  \hypertarget<1>{jumptofirst}{}
  \hypertarget<2>{jumptosecond}{}
  \hypertarget<3>{jumptothird}{}
  \hypertarget<4>{jumptoforth}{}
  \hypertarget<5>{jumptofifth}{}
  \begin{itemize}
  \item<1-> 使用 \textbf{hypertarget} 命令添加链接目标
    \hyperlinkframestartnext{\beamerskipbutton{略过}}
  \item<2-> 使用 \textbf{hyperlink} 命令添加链接跳转
  \item<3->
    \hyperlink{jumptoforth}{\beamergotobutton{到第4页}}
  \item<4->
    \beamerbutton{到第公式(\ref{eq1})}
  \item<5->
    \hyperlink{math<5>}{到数学公式页}
  \item<6->
    \hyperlink{jumptofirst}{\beamerreturnbutton{回第1页}}
  \end{itemize}
}


\subsection{包含图象}
\frame{
  \frametitle{\subsecname}
  \begin{itemize}
  \item 使用graphicx包
    \begin{figure}
      \includegraphics<1->[height=2.5cm,angle=0]{pku-logo.pdf}
      \hfill
      \includegraphics<2->[height=2cm]{pku-tower.pdf}
      \caption{\hfill{博雅塔}}
    \end{figure}

  \item 使用pgfimage命令 \341     \pgfuseimage<3>{logo}
  \end{itemize}
}

\subsection{包含视频和音频}

\frame
{
  \frametitle{\subsecname}
  \begin{itemize}
  \item 视频      \movie[externalviewer,label=mymovie,width=1in,height=0.8in,poster]{\pgfuseimage{logo}}{movie.avi}
%    \hyperlinkmovie[play]{mymovie}{Play}
  \item 声音      \movie[externalviewer,autostart]{这里有一段mp3}{turky.mp3}
  \end{itemize}
}


\frame
{
  \frametitle{完}
  \hypertarget{end}{}
}

\appendix

\section{附录}

\againframe<5>{math}

\frame{ % 由于对包含了verbatim的frame换页有问题,所以我们手工加入一页
  \frametitle{一个Beamer例子文件}
  \beamerEX
}

\clearpage
%\end{CJK*}
\end{document}

